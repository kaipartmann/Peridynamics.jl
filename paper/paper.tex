
% JuliaCon proceedings template
\documentclass{juliacon}
\setcounter{page}{1}

%%---- definition of custom commands
\newcommand{\abs}[1]{\left|#1\right|}
\newcommand{\vb}{\boldsymbol}
%%----

\begin{document}

% **************GENERATED FILE, DO NOT EDIT**************

\title{Simulation of fracture and damage with Peridynamics.jl}

\author[1]{Kai Partmann}
\author[1]{Manuel Dienst}
\author[1]{Kerstin Weinberg}
\affil[1]{Chair of Solid Mechanics, University of Siegen, Siegen, Germany}

\keywords{Julia, Peridynamics, Fracture Mechanics, Simulations, Simulation Framework}

\hypersetup{
pdftitle = {Simulation of fracture and damage with Peridynamics.jl},
pdfsubject = {JuliaCon 2022 Proceedings},
pdfauthor = {Kai Partmann, Manuel Dienst, Kerstin Weinberg},
pdfkeywords = {Julia, Peridynamics, Fracture Mechanics, Simulations, Simulation Framework},
}



\maketitle

\section{Summary}
% from the JOSS docs:
% The paper should include a summary describing the high-level functionality and purpose of the software for a diverse, non-specialist audience.

%=== What is Peridynamics? ===%
\begin{figure}
\centerline{\includegraphics[width=0.8\linewidth]{logo.png}}
\caption{High-velocity contact simulation of three spheres crashing into a rectangular panel; logo of Peridynamics.jl}
\label{fig:logo}
\end{figure}

%- General peridynamics context
Peridynamics is a nonlocal continuum mechanics formulation, which was introduced by Silling \cite{Silling2000}.
It has gained increased popularity as an approach for modeling fracture.
The deformation of the solid is described by integro-differential equations that are also fulfilled for discontinuities, making it very capable of modeling crack propagation and fragmentation with large displacements.
Much peridynamics research has been done in recent years, summarized in various review papers and books \cite{Diehl2019,Javili2019Review,Madenci2014}.

%- Peridynamics short introduction
Typically, in peridynamics the continuum is discretized by material points.
Points interact only with other points inside of their specified \emph{neighborhood} or \emph{point family}~$\mathcal{H}$, which is defined as the set of points inside a sphere with the radius $\delta$, also named the \emph{horizon}.
The interaction of the point $\vb{X}$ with its neighbor $\vb{X}'$ is called \emph{bond} and defined as
\begin{equation}
\vb{\Delta X} = \vb{X}' - \vb{X} \; .
\end{equation}
The equation of motion reads
\begin{equation}
\varrho \, \vb{\ddot{u}}(\vb{X},t) = \vb{b}^{\mathrm{int}}(\vb{X},t) + \vb{b}^{\mathrm{ext}}(\vb{X},t) \; ,
\end{equation} 
with the mass density $\varrho$, the point acceleration vector $\vb{\ddot{u}}$, and the point force density vectors $\vb{b}^{\mathrm{int}}$ and $\vb{b}^{\mathrm{ext}}$.
Various material formulations of peridynamics exist for the calculation of the internal force density $\vb{b}^{\mathrm{int}}$, and all of them are based on the nonlocal interactions between material points.

%- Some information on peridynamic material models
The general internal force density for state-based peridynamics is defined as
\begin{equation}
\vb{b}^{\mathrm{int}} (\vb{X},t) = \int_\mathcal{H} \vb{t} - \vb{t}' \; \mathrm{d}V' \; ,
\end{equation}
with the \emph{force vector states} $\vb{t}=\vb{t}(\vb{\Delta X}, t)$ and $\vb{t}'=\vb{t}(-\vb{\Delta X}, t)$.
In the first original bond-based formulation of peridynamics, the force vector states $\vb{t}$ and  $\vb{t}'$ have the same value and opposite direction.
This implies intrinsic limitation to only one material parameter and in consequence to restrictions on the Poisson's ratio \cite{Silling2007,Trageser2020}.
To overcome these restrictions, a state-based peridynamics was established.
In the ordinary state-based peridynamics, the deformation states of neighboring points also influence the internal force density \cite{Silling2007}.
This leads to force vector states which are still collinear but not of same value anymore.

Further developments are summarized as non-ordinary state-based peridynamics.
A recent development in this regard is continuum-kinematics-inspired peridynamics \cite{Javili2019}.
Another peridynamic formulation is the local continuum consistent correspondence formulation of non-ordinary state-based peridynamics, where an elastic model from the classical local material theory can be used to calculate the internal force density.

\begin{figure}
\centerline{\includegraphics[width=0.9\linewidth]{tensile_test.png}}
\caption{Fracture simulation of tensile tension test with a crack propagating in the middle of the specimen}
\label{fig:tensiletest}
\end{figure}

%=== What is the high-level functionality of Peridynamics.jl? ===%
\texttt{Peridynamics.jl} is an open source Julia \cite{Bezanson2017julia} implementation of peridynamics.
It can be used to conduct simulations with applications such as crack propagation due to external loading conditions (see Fig.~\ref{fig:tensiletest}), or multi-body contact simulations (see Fig.~\ref{fig:logo}).
Users can specify arbitrary geometries as a point cloud to use as a spatial discretization for a simulation.
It is also possible to import meshes generated with ABAQUS and convert them into point clouds.
Multiple peridynamic material models are implemented and can be used in dynamic simulations using Velocity Verlet integration or in quasi-static simulations with an adaptive dynamic relaxation algorithm \cite{Kilic2010}.

%=== What is the purpose of Peridynamics.jl? ===%
The primary purpose of the package is to provide a framework for peridynamic simulations for a broad user base.
A user-friendly interface makes it possible to define geometries and boundary conditions with just a few lines of code.
Due to Julia's multiple dispatch functionality, custom material models can be defined to extend the package.
Therefore, researchers can use the package to develop new peridynamic material models, simplifying their workflow by utilizing the Julia ecosystem.

\section{Statement of need}
% from the JOSS docs:
% The paper should include a Statement of need section that clearly illustrates the research purpose of the software and places it in the context of related work.

%=== Place the package in the context of related work ===%
Various peridynamics software projects exist, including \texttt{PeriPy} \cite{PeriPy2021}, \texttt{NLMech} \cite{Jha2021}, \texttt{Peridigm} \cite{Peridigm2024}, and \texttt{PeriHub} \cite{Willberg2023}.
\texttt{Peridigm} stands out as a mature package capable of large-scale peridynamics simulations that also 
includes several material models and features.
However, the installation process and integrating custom models is notably complex.

The initial intention in developing the package was to simplify the process of implementing new material models within the Julia ecosystem.
In contrast to \texttt{Peridigm}, installing and extending \texttt{Peridynamics.jl} is straightforward, as it benefits from the Julia ecosystem.
\texttt{Peridynamics.jl} was already used in numerous publications \cite{Friebertshaeuser2022PAMM,Friebertshaeuser2022AIMS,Partmann2023IJF,Partmann2024AAM,Partmann2024PAMM,Tornquist2022PAMM}, which shows the practical usability of the package.

\section{Perspectives}
In version \texttt{0.2.0}, a multithreading approach is employed.
An extension to a hybrid approach utilizing MPI and multithreading is currently underway and is planned for version \texttt{0.3.0}.
The internals have been thoroughly reworked to incorporate a parallel data structure, resulting in significantly faster simulations that surpass the performance of \texttt{Peridigm}.
The planned changes include new material models and a simplified API with streamlined internal structures.

\section{Acknowledgments}
The authors gratefully acknowledge the support of the Deutsche Forschungsgemeinschaft (DFG) in the project \mbox{WE~2525/15-1} and \mbox{WE~2525/15-2}.

The authors gratefully acknowledge the support and the supercomputing resources of the Paderborn Center for Parallel Computing (https://pc2.uni-paderborn.de).

% \vadjust{\vfill\pagebreak}

\input{bib.tex}

\end{document}

% Inspired by the International Journal of Computer Applications template
